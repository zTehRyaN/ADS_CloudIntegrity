\section{Class Diagrams}

	Sottolineare il target di tecnologie per cui è effettuata la progettazione. Queste non verranno citate all'interno del capitolo
	di Progetto ma il seguente capitolo di Realizzazione, introducendo banalmente le tecnologie utilizzate, dovrà poter
	calzare perfettamente e dovrà facilmente potersi riallacciare a quanto detto in questo capitolo.

	In questa sezione si ripercorrono gli stessi argomenti della parte di analisi, in parallelo, ma focalizzandosi sulle scelte di
	progetto. Queste dovranno essere motivate e lo scopo di questa porzione di testo è di definire le varie parti e la loro 
	collaborazione. Il livello di dettaglio non sarà massimo in quanto nelle successive sezioni si entrerà nello specifico.

	{Diagrammi} Class diagrams UML o analoghi più efficaci per il comportamento dinamico
	{Immagini}

\section{API}

	{Analisi prettamente tecnologica}
	
	Qui, sulla base di quanto detto nella sezione di analisi, si inizia a mettere insieme le varie parti per descrivere più
	nel dettaglio l'API dal punto di vista progettuale. Già nella prima parte di analisi è stato introdotto a grandi linee il funzionamento.
	
	Il focus è sulle scelte progettuali. Evidenziare dunque i punti di forza e le scelte a livello di pattern.
	
	\subsection{Skiplist Autenticata}
		
		Qui si entra in dettaglio sulla struttura principale, autenticata. Quindi vengono analizzate più nel dettaglio alcune scelte progettuali
		su questa struttura
	
	\subsection{Schema hashing}
	
		Breve sezione che riprende l'argomento dell'hashing e che lo analizza concentrandosi sulle scelte progettuali.
		
	\subsection{Proof}
	
%		MODELLO FUNZIONALE, rimandando a parte successiva 
%		
%		Qui si entra in dettaglio sulla struttura secondaria, con dettagli di autenticazione. Quindi vengono analizzate più nel dettaglio alcune scelte progettuali
%		su questa struttura
	
\section{Persistenza}

	Introduzione al discorso sulla persistenza, riallacciandosi a ciò che è stato detto nella parte di analisi.
	
	VISTO CHE NON SARA' PRESENTE UNA PARTE DI REALIZZAZIONE PER LA PARTE DI PERSISTENZA, E DUNQUE
	NON SARA' POSSIBILE PARLARE DELLA TECNOLOGIE UTILIZZATE, CERCARE DI SOTTOLINEARLO INDIRETTAMENTE QUI,
	PUR SEMPRE APPUNTO NON CITANDO ESATTAMENTE LE TECNOLOGIE CHE SI E' PENSATO DI APPLICARE.
		
	\subsection{Studi teorici}
		
		Qui viene affrontato il problema dal punto di vista teorico, con particolare attenzione a sottolineare i problemi teorici derivati dallo studio 
		teorico operato per quanto riguarda la persistenza di una struttura dati su base di dati NoSQL. Qui darei un accento più teorico, mentre le 
		scelte progettuali che ne possono derivare, le descriverei nelle successive sezioni.
		
	\subsection{Progettazione}
	
	Questa è la sezione in cui si parla di tutto il progetto legato alla persistenza, e di come questo si lega al tutto.
	E' importante qui accennare tutti gli attori e la loro interazione globale. Il focus è sulle scelte progettuali, e sui
	pattern applicati, ma saranno descritti al massimo livello di dettaglio solo nelle sottosezioni successive.
		
	\subsection{Translator}
	
		Entro nel dettaglio del Translator, descrivendo compiti e giustificandolo tramite pattern
		
	\subsection{Connector}
	
		Entro nel dettaglio del Connector, descrivendo compiti e giustificandolo tramite pattern
	
	\subsection{CassandraService}
	
		Entro nel dettaglio del Service, descrivendo compiti e giustificandolo tramite pattern.
		Enfasi su come questa parte sia un centro-stella per le altre parti progettuali
		
	\subsection{Comportamento asincrono}
	
		In base a quanto detto, sottolineare le decisioni progettuali e di come queste possano essere ottimizzate con
		comportamento asincrono. Sottolineare bene che uno degli obiettivi è stato quello di minimizzare i round di 
		query con Cassandra, e che questo è reso possibile da coroutine tipiche di linguaggi di programmazione OO moderni.
			
	\subsection{Caching}
	
		In questa parte si analizza il problema di caching con relativi problemi di "always-on". Descrivere compiti e giustificazioni
		tramite pattern e future proofing. Giustificare anceh in funzione del contesto piu generale delle dimensioni dell'ADS (fare calcolo su stime)