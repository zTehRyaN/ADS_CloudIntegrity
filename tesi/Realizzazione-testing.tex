\section{Realizzazione}

%	In questa parte è bene marcare ciò che è stato effettivamente realizzato. Le scelte progettuali sono state descritte
%	in precedenza, ma qui si parla della loro applicazione. Qui dunque si entra nello specifico.

\section{Testing}

%	Si parla del TDD. E di come è stata condotta la programmazione: a cavallo tra il TDD e il DataDD.
%	Il testing è stato fatto anche in virtù dell'utilizzo dell'API, simulando la concorrenza tramite thread di sistema

%	[Appunti su testing]
	
	[REALIZZARE TEST CON THREAD]

\section{Analisi Prestazioni}

%	Parte opzionale, in cui si introduce all'analisi di prestazioni effettuata.
	
	\subsection{Schema hashing}
	
%		Si parla qui di alcuni parametri che possono influire sull'analisi prestazionali. Analisi dunque indipendente e isolata.
		
	\subsection{Curve fitting}
	
%		Si parla qui nello specifico del Curve Fitting e di come questo sia stato applicato. Risultati raggiunti ed eventualmente 
%		grafici autogenerati dall'esecuzione del programma e/o generati a posteriori.