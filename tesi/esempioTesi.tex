\documentclass{TesiDiaUniroma3}

\titolo{Analisi e progettazione di strutture dati autenticate in memoria e persistenti}
\autore{Matteo Giordano}
\matricola{499253}
\relatore{Prof. Maurizio Pizzonia}
\correlatore{Diego Pennino} % modifica anche TesiDiaUniroma3.cls se vuoi avere un correlatore
\annoAccademico{2017/2018}

% dati opzionali
\dedica{Questa è la dedica} % solo se nel documento si usa il comando \generaDedica

\usepackage[plainpages=false]{hyperref}	% generazione di collegamenti ipertestuali su indice e riferimenti
\usepackage[all]{hypcap} % per far si che i link ipertestuali alle immagini puntino all'inizio delle stesse e non alla didascalia sottostante
\usepackage{amsthm}	% per definizioni e teoremi
\usepackage[italian]{babel}
\usepackage{graphicx}
\usepackage{lmodern}
\usepackage[T1]{fontenc}
\usepackage[utf8]{inputenc}
\usepackage[]{algorithm2e}
\usepackage{amsmath}

\begin{document}
\frontmatter
\generaFrontespizio
\generaDedica
\ringraziamenti{ringraziamenti}	% inserisce i ringraziamenti e li prende in questo caso da ringraziamenti.tex
\introduzione{introduzione}		% inserisce l'introduzione e la prende in questo caso da introduzione.tex
\generaIndice
\generaIndiceFigure


\mainmatter
\capitolo{Stato dell'arte}{StatoArte}
\capitolo{Analisi}{Analisi}
\capitolo{Progetto}{Progetto}
\capitolo{Realizzazione e testing}{Realizzazione-testing}


\backmatter
\conclusioni{conclusioni}

\bibliography{bibliografia} % inserisce la bibliografia e la prende in questo caso da bibliografia.bib

\end{document}
