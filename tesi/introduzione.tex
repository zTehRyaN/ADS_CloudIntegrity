\section{Contestualizzazione}
	
%	In questa sezione viene definito il contesto in cui si trova il progetto e l'esperimento svolto.
%	Viene dunque descritto l'ambito in cui è definito il tutto.
%	
%	Quindi ambito Cloud, Trusted/Untrusted, integrità dei dati, storage illimitato ma non fidato nel cloud e storage molto limitato ma fidato "in memory"
	 
	\subsection{Definizione dei problemi affrontati}
	
%		In questa sottosezione si va più nello specifico descrivendo i problemi affrontati all'interno del contesto 
%		precedentemente descritto.
%		Perciò si introduce ai problemi relativi ai punti della sezione precedente.

\section{Obiettivi}

%	A questo punto si descrivono gli obiettivi, in funzione dei problemi affrontati. Sia obiettivi generali, di superamento dei problemi 
%	precedentemente descritti e in seguito accenni a obiettivi di cui si è tenuto conto in fase di analisi, progetto e realizzazione.
%	()Magari confermati da fase di testing?)

\section{Conclusione}
	
	\subsection{Risultati raggiunti}
	
%		Viene qui brevemente accennato ai risultati raggiunti facendo però molta attenzione a sottolinearne il valore 
		
	\subsection{Aspetti interessanti del lavoro}
	
%	Utile sezione per poter sottolineare meglio punti critici e di particolare interesse algoritmico concettuale o magari implementativo
	
	\subsection{Struttura della tesi}
	
%	In questa sottosezione si illustra la modalità con cui è stata strutturata la tesi. E' una breve descrizione discorsiva che ripercorre tutte le sezioni e ne
%	definisce bene concatenazione causale e collegamenti tra le parti.